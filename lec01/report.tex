\documentclass[a4paper,12pt]{article}
\usepackage[T1]{fontenc}
\usepackage[utf8]{inputenc}
\usepackage{polski}
\usepackage{color}
\usepackage{graphicx}
\usepackage{amsmath}
\usepackage{amssymb}
\usepackage{hyperref}
\usepackage{float}
\usepackage{listings}
\usepackage[backend=biber]{biblatex}

\addbibresource{bibliografia.bib}

\lstset{
  basicstyle=\ttfamily,
  columns=flexible,
  keepspaces=true,
  literate={ą}{{\k{a}}}1
           {ć}{{\'{c}}}1
           {ę}{{\k{e}}}1
           {ł}{{\l{}}}1
           {ń}{{\'{n}}}1
           {ó}{{\'{o}}}1
           {ś}{{\'{s}}}1
           {ź}{{\'{z}}}1
           {ż}{{\.{z}}}1
           {Ą}{{\k{A}}}1
           {Ć}{{\'{C}}}1
           {Ę}{{\k{E}}}1
           {Ł}{{\L{}}}1
           {Ń}{{\'{N}}}1
           {Ó}{{\'{O}}}1
           {Ś}{{\'{S}}}1
           {Ź}{{\'{Z}}}1
           {Ż}{{\.{Z}}}1
}

\title{Zadanie - wykład 1}
\author{Mikołaj Kubś, 272662}
\date{\today}

\begin{document}

\maketitle

\begin{figure}[H]
    \centering
    \includegraphics[width=1\textwidth]{images/task.png}
    \caption{Opis zadania}
\end{figure}

\section{Cele biznesowe}

Na podstawie danych zawartych w hurtowni AdventureWorksDW można realizować następujące cele biznesowe:

\begin{enumerate}
    \item \textbf{Analiza sprzedaży} - monitorowanie trendów sprzedaży w różnych okresach i kategoriach produktów.
    \item \textbf{Segmentacja klientów} - identyfikacja najważniejszych grup klientów na podstawie geolokalizacji i historii zakupów.
    \item \textbf{Optymalizacja promocji} - ocena skuteczności różnych promocji na podstawie wyników sprzedaży.
    \item \textbf{Zarządzanie zapasami} - prognozowanie popytu na podstawie wcześniejszych zamówień i sezonowości.
    \item \textbf{Analiza regionalna} - badanie wyników sprzedaży w różnych regionach w celu optymalizacji działań marketingowych.
    \item \textbf{Ocena rentowności produktów} - identyfikacja najbardziej i najmniej dochodowych produktów.
    \item \textbf{Analiza terminowości dostaw} - ocena terminowości realizacji zamówień i identyfikacja problemów logistycznych.
    \item \textbf{Wspomaganie decyzji cenowych} - analiza wpływu cen na sprzedaż różnych produktów.
    \item \textbf{Analiza walutowa} - ocena wpływu kursów walut na wyniki sprzedaży międzynarodowej.
    \item \textbf{Optymalizacja kanałów sprzedaży} - identyfikacja najlepszych kanałów sprzedaży i optymalizacja strategii marketingowej.
\end{enumerate}

\section{Możliwe analizy}

Na podstawie danych z hurtowni AdventureWorksDW można przeprowadzić następujące analizy:

\begin{enumerate}
    \item \textbf{Sprzedaż produktów w czasie} - analiza sprzedaży w różnych okresach (np. miesiąc, kwartał, rok).
    \item \textbf{Skuteczność promocji} - porównanie sprzedaży produktów objętych promocją i tych, które nie były promowane.
    \item \textbf{Sprzedaż w podziale na regiony} - analiza sprzedaży w różnych krajach/terytoriach.
    \item \textbf{Segmentacja klientów według przychodów} - grupowanie klientów na podstawie wartości ich zamówień.
    \item \textbf{Wpływ ceny na sprzedaż} - analiza korelacji między ceną produktu a jego sprzedażą.
    \item \textbf{Czas dostawy a zadowolenie klientów} - badanie zależności między czasem realizacji zamówienia a liczbą zwrotów lub reklamacji.
    \item \textbf{Najlepiej sprzedające się produkty} - ranking produktów według liczby sprzedanych sztuk i wartości sprzedaży.
    \item \textbf{Sprzedaż w podziale na kategorie produktów} - porównanie sprzedaży różnych typów produktów.
    \item \textbf{Wpływ waluty na sprzedaż międzynarodową} - analiza sprzedaży w zależności od kursów walutowych.
    \item \textbf{Liczba nowych klientów w czasie} - analiza wzrostu bazy klientów w określonych przedziałach czasowych.
\end{enumerate}

\end{document}