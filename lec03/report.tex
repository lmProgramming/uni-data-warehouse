\documentclass[a4paper,12pt]{article}
\usepackage[T1]{fontenc}
\usepackage[utf8]{inputenc}
\usepackage{polski}
\usepackage{color}
\usepackage{graphicx}
\usepackage{amsmath}
\usepackage{amssymb}
\usepackage{hyperref}
\usepackage{float}
\usepackage{listings}
\usepackage[backend=biber]{biblatex}

\addbibresource{bibliografia.bib}

\lstset{
  basicstyle=\ttfamily,
  columns=flexible,
  keepspaces=true,
  showstringspaces=false,
  escapeinside={(*@}{@*)},
  literate={ą}{{\k{a}}}1
           {ć}{{\'{c}}}1
           {ę}{{\k{e}}}1
           {ł}{{\l{}}}1
           {ń}{{\'{n}}}1
           {ó}{{\'{o}}}1
           {ś}{{\'{s}}}1
           {ź}{{\'{z}}}1
           {ż}{{\.{z}}}1
           {Ą}{{\k{A}}}1
           {Ć}{{\'{C}}}1
           {Ę}{{\k{E}}}1
           {Ł}{{\L{}}}1
           {Ń}{{\'{N}}}1
           {Ó}{{\'{O}}}1
           {Ś}{{\'{S}}}1
           {Ź}{{\'{Z}}}1
           {Ż}{{\.{Z}}}1
           {"}{{\textquotedbl}}1
           {'}{{\textquotesingle}}1
           {`}{{\textasciigrave}}1
           {~}{{\textasciitilde}}1
           {^}{{\textasciicircum}}1
           {_}{{\textunderscore}}1
           {|}{{\textbar}}1
           {\{}{{\textbraceleft}}1
           {\}}{{\textbraceright}}1
           {[}{{[}}1
           {]}{{]}}1
}

\title{Zadanie - wykład 3}
\author{Mikołaj Kubś, 272662}
\date{\today}

\begin{document}

\maketitle

\begin{figure}[H]
	\centering
	\includegraphics[width=1\textwidth]{images/task1.png}
	\caption{Opis zadań 1-3}
\end{figure}

\begin{figure}[H]
	\centering
	\includegraphics[width=1\textwidth]{images/task2.png}
	\caption{Opis zadań 4-6}
\end{figure}

\section{Kod kwerend 1-3}

\subsection{Kwerenda 1}

"Jaka była łączna suma transakcji (SalesOrderHeader.SubTotal) w poszczególnych latach dla kolejnych dni tygodnia?"

\begin{lstlisting}[
  language=SQL,
  showspaces=false,
  basicstyle=\ttfamily,
  numbers=left,
  numberstyle=\tiny,
  commentstyle=\color{green},
  tabsize=2
  ]
SET DATEFIRST 1;
SET LANGUAGE Polish;

SELECT 
  SUM(SubTotal) AS "Suma",
  DATENAME(DW, OrderDate) AS "Dzień tygodnia",
  DATEPART(YEAR, OrderDate) AS "Rok"
FROM Sales.SalesOrderHeader
GROUP BY 
  DATENAME(DW, OrderDate), 
  DATEPART(YEAR, OrderDate), 
  DATEPART(DW, OrderDate)
ORDER BY DATEPART(YEAR, OrderDate), DATEPART(DW, OrderDate)
\end{lstlisting}

\begin{figure}[H]
	\centering
	\includegraphics[width=1.0\textwidth]{images/1.png}
	\caption{Wyniki 1. kwerendy}
\end{figure}

W tym podpunkcie użycie CTE nie przyniosłoby korzyści.

\begin{figure}[H]
	\centering
	\includegraphics[width=0.4\textwidth]{images/1_excel.png}
	\caption{Tabela przestawna dla 1. kwerendy}
\end{figure}

Dla łatwiejszej analizy dodano tabelę przestawną.

W 2012 i 2013 roku suma roczna szybko rosła. 2014 rok się jeszcze nie skończył, co częściowo wyjaśnia niższą sumę dla 2014 roku.

W 2011 roku przychody w niedzielę były 2 razy niższe niż w drugim najgorszym dniu tygodnia tego roku. Poniedziałek był zdecydowanie najlepszy, a sobota druga.

W 2012 roku piątek był dniem bardzo niskich przychodów - suma była ponad 16 razy mniejsza niż dla najlepszego dnia, niedzieli.

W 2013 roku przychody były bardziej wyrównany, gdzie piątek (najsłabszy dzień) był ponad 2 razy mniej dochodowy niż najlepszy (środa).

W 2014 roku wtorek osiągnął bardzo niski wynik, prawie 40 razy niższy od najlepszego (czwartku).

Wnioskiem jest to, że różne dni w różnych latach odbiegają od normy dla danego roku. Jednak nie w każdym roku różnice były aż tak widoczne.

\subsection{Kwerenda 2}

"Zaproponuj podział klientów na 3 rozłączne grupy wiekowe. Ilu różnych klientów dokonało zakupów w kolejnych miesiącach roku w każdej z grup? Ilu klientów w poszczególnych grupach wykonało zakup dokładnie jeden raz?"

Zdecydowano się na podział klientów na 3 rozłączne grupy wiekowe o możliwie najrówniejszej liczbie członków. Do zrealizowania tego użyto funkcji "NTILE". Grupa 1 to najmłodsi, a 3 to najstarsi.

Zadanie dotyczy tak naprawdę dwóch kwerend.

\subsubsection{Ilu różnych klientów dokonało zakupów w kolejnych miesiącach roku w każdej z grup?}

Wersja z CTE:

{\small
\begin{lstlisting}[
	language=SQL,
	showspaces=false,
	basicstyle=\ttfamily,
	numbers=left,
	numberstyle=\tiny,
	commentstyle=\color{green},
	tabsize=2
]
WITH CustomerOrders AS (
	SELECT 
		Customer.CustomerID,
		NTILE(3) OVER (ORDER BY YEAR(GETDATE()) - YEAR(
			Person.Demographics.value(
				'declare default element namespace 
"http://schemas.microsoft.com/sqlserver/2004/07/adventure-works/IndividualSurvey";
(//BirthDate)[1]', 
				'DATE'
			)
		)) AS AgeGroup,
		YEAR(SalesOrderHeader.OrderDate) AS OrderYear
		MONTH(SalesOrderHeader.OrderDate) AS OrderMonth,
	FROM Sales.Customer AS Customer
	JOIN Person.Person AS Person ON Person.BusinessEntityID = Customer.PersonID
	JOIN 
		Sales.SalesOrderHeader AS SalesOrderHeader ON
			SalesOrderHeader.CustomerID = Customer.CustomerID
	WHERE
		Person.Demographics.exist(
			'declare default element namespace 
"http://schemas.microsoft.com/sqlserver/2004/07/adventure-works/IndividualSurvey";
(//BirthDate)[1]') = 1
)
SELECT
	OrderYear AS "Rok",
	OrderMonth AS "Miesiąc",
	AgeGroup AS "Grupa wiekowa", 
	COUNT(DISTINCT CustomerID) AS "Liczba unikalnych klientów"
FROM CustomerOrders
GROUP BY AgeGroup, OrderYear, OrderMonth
ORDER BY OrderYear, OrderMonth, AgeGroup
\end{lstlisting}}

Wersja bez CTE:

{\small
\begin{lstlisting}[
	language=SQL,
	showspaces=false,
	basicstyle=\ttfamily,
	numbers=left,
	numberstyle=\tiny,
	commentstyle=\color{green},
	tabsize=2
]
SELECT
    AgeGroup AS "Grupa wiekowa",
    OrderYear AS "Rok",
    OrderMonth AS "Miesiąc",
    COUNT(DISTINCT CustomerID) AS "Liczba unikalnych klientów"
FROM (
    SELECT 
        Customer.CustomerID,
        NTILE(3) OVER (ORDER BY YEAR(GETDATE()) - YEAR(
            Person.Demographics.value(
                'declare default element namespace 
"http://schemas.microsoft.com/sqlserver/2004/07/adventure-works/IndividualSurvey";
                (//BirthDate)[1]', 
                'DATE'
            )
        )) AS AgeGroup,
        YEAR(SalesOrderHeader.OrderDate) AS OrderYear,
        MONTH(SalesOrderHeader.OrderDate) AS OrderMonth
    FROM Sales.Customer AS Customer
    INNER JOIN Person.Person AS Person 
        ON Customer.PersonID = Person.BusinessEntityID
    INNER JOIN Sales.SalesOrderHeader AS SalesOrderHeader 
        ON Customer.CustomerID = SalesOrderHeader.CustomerID
    WHERE Person.Demographics.exist(
        'declare default element namespace 
"http://schemas.microsoft.com/sqlserver/2004/07/adventure-works/IndividualSurvey";
        (//BirthDate)[1]') = 1
) AS CustomerOrderData
GROUP BY AgeGroup, OrderYear, OrderMonth
ORDER BY OrderYear, OrderMonth, AgeGroup
\end{lstlisting}}

\begin{figure}[H]
	\centering
	\includegraphics[width=1.0\textwidth]{images/2.png}
	\caption{Wyniki 2. kwerendy}
\end{figure}

Wersja "bez" CTE i z nim są bardzo podobne. Nie da się przenieść bezpośrednio NTILE'a i reszty logiki do jednego zapytania bez wcześniejszego wyliczenia grup wiekowych, ponieważ nie da się pogrupować po funkcji okienkowej. Nielogiczne byłoby naraz grupowanie i wybieranie. Jednym ze sposobów rozwiązania tego problemu mogłoby być wyliczenie wcześniej górnej granicy wieku dla 1. i 2. grupy, a następnie przypisanie klientów i grupowanie w jednym zapytanie - nadal trzeba jednak coś wyliczyć wcześniej.

W większości miesięcy liczba unikalnych klientów jest względnie podobna do siebie dla każdej grupy. Zdarza się, że najliczniejsza grupa jest 2-3 razy liczniejsza od tej najmniej licznej w danym miesiącu. Często się potem w tym samym roku zdarza odwrotna sytuacja. W 2012 roku najmłodsza grupa jest najliczniejsza, a w 2014 najmniej liczna o dość podobną liczbę osób.

\begin{figure}[H]
	\centering
	\includegraphics[width=1.0\textwidth]{images/2_execution_plan.png}
	\caption{Porównanie execution plan}
\end{figure}

Obie kwerendy działają tak samo z wcześniej opisanych powodów.

\subsubsection{Ilu klientów w poszczególnych grupach wykonało zakup dokładnie jeden raz?}

Wersja z CTE:

{\small
\begin{lstlisting}[
	language=SQL,
	showspaces=false,
	basicstyle=\ttfamily,
	numbers=left,
	numberstyle=\tiny,
	commentstyle=\color{green},
	tabsize=2
]
WITH OrdersWithCustomerDetails AS (
    SELECT 
        Customer.CustomerID,
        NTILE(3) OVER (ORDER BY YEAR(GETDATE()) - YEAR(
            Person.Demographics.value(
                'declare default element namespace 
"http://schemas.microsoft.com/sqlserver/2004/07/adventure-works/IndividualSurvey";
                (//BirthDate)[1]', 
                'DATE'
            )
        )) AS AgeGroup,
        SalesOrderHeader.SalesOrderID,
        YEAR(SalesOrderHeader.OrderDate) AS OrderYear,
        MONTH(SalesOrderHeader.OrderDate) AS OrderMonth
    FROM Sales.Customer AS Customer
    INNER JOIN Person.Person AS Person
        ON Customer.PersonID = Person.BusinessEntityID
    INNER JOIN Sales.SalesOrderHeader AS SalesOrderHeader
        ON Customer.CustomerID = SalesOrderHeader.CustomerID
    WHERE Person.Demographics.exist(
        'declare default element namespace 
"http://schemas.microsoft.com/sqlserver/2004/07/adventure-works/IndividualSurvey";
        (//BirthDate)[1]') = 1
),
CustomersWithSingleOrder AS (
    SELECT
        CustomerID,
        AgeGroup
    FROM OrdersWithCustomerDetails
    GROUP BY CustomerID, AgeGroup
    HAVING COUNT(SalesOrderID) = 1
)
SELECT
    AgeGroup AS "Grupa wiekowa",
    COUNT(CustomerID) AS "Liczba klientów z pojedynczym zamówieniem"
FROM CustomersWithSingleOrder
GROUP BY AgeGroup
ORDER BY AgeGroup
\end{lstlisting}}

Wersja bez CTE:

{\small
\begin{lstlisting}[
	language=SQL,
	showspaces=false,
	basicstyle=\ttfamily,
	numbers=left,
	numberstyle=\tiny,
	commentstyle=\color{green},
	tabsize=2
]
SELECT
    CustomerAgeGroup AS "Grupa wiekowa",
    COUNT(*) AS "Liczba klientów z jednym zamówieniem"
FROM (
    SELECT 
        CustomerID,
        CustomerAgeGroup
    FROM (
        SELECT 
            SalesCustomer.CustomerID,
            NTILE(3) OVER (ORDER BY YEAR(GETDATE()) - YEAR(
                PersonTable.Demographics.value(
                    'declare default element namespace 
"http://schemas.microsoft.com/sqlserver/2004/07/adventure-works/IndividualSurvey";
                    (//BirthDate)[1]', 
                    'DATE'
                )
            )) AS CustomerAgeGroup,
            SalesOrderHeader.SalesOrderID
        FROM Sales.Customer AS SalesCustomer
        INNER JOIN Person.Person AS PersonTable 
            ON SalesCustomer.PersonID = PersonTable.BusinessEntityID
        INNER JOIN Sales.SalesOrderHeader AS SalesOrderHeader 
            ON SalesCustomer.CustomerID = SalesOrderHeader.CustomerID
        WHERE PersonTable.Demographics.exist(
            'declare default element namespace 
"http://schemas.microsoft.com/sqlserver/2004/07/adventure-works/IndividualSurvey";
            (//BirthDate)[1]'
        ) = 1
    ) AS OrdersPerCustomer
    GROUP BY CustomerID, CustomerAgeGroup
    HAVING COUNT(SalesOrderID) = 1
) AS SingleOrderCustomers
GROUP BY CustomerAgeGroup
ORDER BY CustomerAgeGroup
\end{lstlisting}}

\begin{figure}[H]
	\centering
	\includegraphics[width=1.0\textwidth]{images/2.1.png}
	\caption{Wyniki 2. kwerendy, część 2.}
\end{figure}

Wersja "bez" CTE i z CTE są bardzo podobne, z powodów opisanych wcześniej, dla pierwszej kwerendy w zadaniu 2. Dużym błędem byłoby tutaj wyliczenie najpierw klientów z jedną transakcją, a potem grupowanie ich NTILE'm na podstawie wyniku - NTILE wtedy analizowałby tylko wycinek całej populacji

Liczba klientów z jedną transakcją jest względnie podobna dla każdej grupy klientów. Najwięcej najstarszych klientów kupiło coś tylko raz. Najmniej młodych dokonało zakupu tylko raz.

\begin{figure}[H]
	\centering
	\includegraphics[width=1.0\textwidth]{images/2_execution_plan.png}
	\caption{Porównanie execution plan}
\end{figure}

Obie kwerendy działają tak samo z wcześniej opisanych powodów.

\subsection{Kwerenda 3}

"Przygotuj zestawienie produktów, których sprzedaje się miesięcznie min. 20 sztuk. Dla każdego produktu podaj jego kategorię"

Kwerenda z CTE:

{\small
\begin{lstlisting}[
	language=SQL,
	showspaces=false,
	basicstyle=\ttfamily,
	numbers=left,
	numberstyle=\tiny,
	commentstyle=\color{green},
	tabsize=2
]
WITH ProductCategories AS (
	SELECT 
		ProductSubcategory.ProductSubcategoryID,
		ProductCategory.Name AS CategoryName
	FROM Production.ProductSubcategory
	JOIN Production.ProductCategory 
		ON ProductSubcategory.ProductCategoryID = ProductCategory.ProductCategoryID
),
MonthlySalesWithMin AS (
	SELECT 
		Product.ProductID,
		Product.ProductSubcategoryID,
		Product.Name AS ProductName,
		YEAR(SalesOrderHeader.OrderDate) AS SalesYear,
		MONTH(SalesOrderHeader.OrderDate) AS SalesMonth,
		SUM(SalesDetail.OrderQty) AS MonthlyQty,
		MIN(SUM(SalesDetail.OrderQty)) OVER (PARTITION BY Product.ProductID) 
			AS MinMonthlyQty
	FROM Sales.SalesOrderDetail AS SalesDetail
	JOIN Production.Product AS Product 
		ON SalesDetail.ProductID = Product.ProductID
	JOIN Sales.SalesOrderHeader AS SalesOrderHeader
		ON SalesDetail.SalesOrderID = SalesOrderHeader.SalesOrderID
	GROUP BY 
		Product.ProductID, 
		Product.ProductSubcategoryID,
		Product.Name, 
		YEAR(SalesOrderHeader.OrderDate), 
		MONTH(SalesOrderHeader.OrderDate)
)
SELECT 
	MonthlySalesWithMin.ProductName AS "Nazwa produktu",
	ProductCategories.CategoryName AS "Nazwa kategorii",
	MonthlySalesWithMin.SalesYear AS "Rok",
	MonthlySalesWithMin.SalesMonth AS "Miesiąc",
	MonthlySalesWithMin.MonthlyQty AS "Liczba zakupionych produktów"
FROM MonthlySalesWithMin
LEFT JOIN ProductCategories 
	ON MonthlySalesWithMin.ProductSubcategoryID = 
		ProductCategories.ProductSubcategoryID
WHERE MonthlySalesWithMin.MinMonthlyQty >= 20
ORDER BY 
	MonthlySalesWithMin.ProductName, 
	MonthlySalesWithMin.SalesYear, 
	MonthlySalesWithMin.SalesMonth
\end{lstlisting}}

Kwerenda bez CTE:

{\small
\begin{lstlisting}[
	language=SQL,
	showspaces=false,
	basicstyle=\ttfamily,
	numbers=left,
	numberstyle=\tiny,
	commentstyle=\color{green},
	tabsize=2
]
SELECT 
	Sales.ProductName AS "Nazwa produktu",
	Sales.CategoryName AS "Nazwa kategorii",
	Sales.SalesYear AS "Rok",
	Sales.SalesMonth AS "Miesiąc",
	Sales.MonthlyQty AS "Liczba zakupionych produktów"
FROM (
	SELECT 
		Product.ProductID,
		Product.Name AS ProductName,
		ProductCategory.Name AS CategoryName,
		YEAR(SalesOrderHeader.OrderDate) AS SalesYear,
		MONTH(SalesOrderHeader.OrderDate) AS SalesMonth,
		SUM(SalesDetail.OrderQty) AS MonthlyQty
	FROM Sales.SalesOrderDetail AS SalesDetail
	JOIN Production.Product AS Product 
		ON SalesDetail.ProductID = Product.ProductID
	JOIN Sales.SalesOrderHeader AS SalesOrderHeader
		ON SalesDetail.SalesOrderID = SalesOrderHeader.SalesOrderID
	LEFT JOIN Production.ProductSubcategory AS ProductSubcategory
		ON Product.ProductSubcategoryID = ProductSubcategory.ProductSubcategoryID
	LEFT JOIN Production.ProductCategory AS ProductCategory
		ON ProductSubcategory.ProductCategoryID = ProductCategory.ProductCategoryID
	GROUP BY 
		Product.ProductID, 
		Product.Name,
		ProductCategory.Name,
		YEAR(SalesOrderHeader.OrderDate), 
		MONTH(SalesOrderHeader.OrderDate)
) AS Sales
INNER JOIN (
	SELECT ProductID
	FROM (
		SELECT 
			Product.ProductID,
			SUM(SalesDetail.OrderQty) AS MonthlyQty
		FROM Sales.SalesOrderDetail AS SalesDetail
		JOIN Production.Product AS Product 
			ON SalesDetail.ProductID = Product.ProductID
		JOIN Sales.SalesOrderHeader AS SalesOrderHeader
			ON SalesDetail.SalesOrderID = 
				SalesOrderHeader.SalesOrderID
		GROUP BY 
			Product.ProductID,
			YEAR(SalesOrderHeader.OrderDate), 
			MONTH(SalesOrderHeader.OrderDate)
	) AS Monthly
	GROUP BY ProductID
	HAVING MIN(MonthlyQty) >= 20
) AS FilteredProducts
    ON Sales.ProductID = FilteredProducts.ProductID
ORDER BY Sales.ProductName, Sales.SalesYear, Sales.SalesMonth;
\end{lstlisting}}

\begin{figure}[H]
	\centering
	\includegraphics[width=1.0\textwidth]{images/3.png}
	\caption{Wyniki 3. kwerendy}
\end{figure}

\begin{figure}[H]
	\centering
	\includegraphics[width=0.25\textwidth]{images/3_excel.png}
	\caption{Tabela przestawna}
\end{figure}

Są 52 produkty, które konsekwentnie sprzedają się w liczbie co najmniej 20 sztuk miesięcznie. Z podziałem na kategorie, to rowery i komponenty są najbardziej konsekwentnie kupowane, potem ubrania i akcesoria.

Trzeba pamiętać, że kwerenda z tak sztywnymi zasadami mogła wykluczyć wiele dobrze sprzedających się produktów, które mogły sprzedać się w niskiej liczbie w co najmniej jednym miesiącu z wielu.

\begin{figure}[H]
	\centering
	\includegraphics[width=1.0\textwidth]{images/3_execution_plan.png}
	\caption{Porównanie exeuction plan}
\end{figure}

Wersja z CTE jest tylko nieznacznie szybsza - 46\% do 54\%.

\section{Kod kwerend 4-6}

\subsection{Kwerenda 4}

"Przygotuj zestawienie, w którym przeanalizujesz, ilu jest różnych klientów dla każdej płci w kolejnych miesiącach (05.2011 - 06.2024)?

Jak procentowo rozkłada się ich udział w całkowitej wartości sprzedaży (Sales.SalesOrderHeader.TotalDue)?"

{\small
\begin{lstlisting}[
	language=SQL,
	showspaces=false,
	basicstyle=\ttfamily,
	numbers=left,
	numberstyle=\tiny,
	commentstyle=\color{green},
	tabsize=2
]
WITH CustomersGender AS (
	SELECT Customer.CustomerID, 
		Demographics.value(
			'declare default element namespace 
"http://schemas.microsoft.com/sqlserver/2004/07/adventure-works/IndividualSurvey"; 
			(//Gender)[1]', 
			'CHAR(1)'
		) AS Gender
	FROM Sales.Customer AS Customer
	INNER JOIN Person.Person AS Person ON Person.BusinessEntityID = Customer.PersonID
	WHERE Demographics.exist(
		'declare default element namespace 
"http://schemas.microsoft.com/sqlserver/2004/07/adventure-works/IndividualSurvey"; 
		(//Gender)[1]'
	) = 1
),
SalesData AS (
	SELECT 
		YEAR(SalesOrderHeader.OrderDate) AS Year, 
		MONTH(SalesOrderHeader.OrderDate) AS Month, 
		CustomersGender.Gender,
		SUM(SalesOrderHeader.TotalDue) AS TotalDue,
		COUNT(DISTINCT CustomersGender.CustomerID) AS NumberOfClients
	FROM Sales.SalesOrderHeader AS SalesOrderHeader
	INNER JOIN CustomersGender ON 
		CustomersGender.CustomerID = SalesOrderHeader.CustomerID
	WHERE SalesOrderHeader.OrderDate BETWEEN 
		'2011-05-01' AND '2024-06-30'
	GROUP BY 
		YEAR(SalesOrderHeader.OrderDate), 
		MONTH(SalesOrderHeader.OrderDate), 
		CustomersGender.Gender
),
TotalSalesPerPeriod AS (
	SELECT Year, Month, SUM(TotalDue) AS TotalDueAll
	FROM SalesData
	GROUP BY Year, Month
)
SELECT 
	SalesData.Year AS "Rok",
	SalesData.Month AS "Miesiąc",
	SUM(CASE WHEN SalesData.Gender = 'M' THEN 
		SalesData.TotalDue ELSE 0 END) AS "Wartość sprzedaży M",
	SUM(CASE WHEN SalesData.Gender = 'F' THEN 
		SalesData.TotalDue ELSE 0 END) AS "Wartość sprzedaży K",
	SUM(CASE WHEN SalesData.Gender = 'M' THEN 
		SalesData.NumberOfClients ELSE 0 END) AS "Liczba klientów",
	SUM(CASE WHEN SalesData.Gender = 'F' THEN 
		SalesData.NumberOfClients ELSE 0 END) AS "Liczba klientek",
	ROUND(CAST(SUM(CASE WHEN SalesData.Gender = 'M' THEN 
		SalesData.TotalDue ELSE 0 END) AS FLOAT) / TotalSalesPerPeriod.TotalDueAll, 2) 
		AS "Udział M",
	ROUND(CAST(SUM(CASE WHEN SalesData.Gender = 'F' THEN 
		SalesData.TotalDue ELSE 0 END) AS FLOAT) / TotalSalesPerPeriod.TotalDueAll, 2)
		AS "Udział K"
FROM SalesData
INNER JOIN TotalSalesPerPeriod ON 
	SalesData.Year = TotalSalesPerPeriod.Year AND 
	SalesData.Month = TotalSalesPerPeriod.Month
GROUP BY SalesData.Year, SalesData.Month, TotalSalesPerPeriod.TotalDueAll
ORDER BY "Rok", "Miesiąc"
\end{lstlisting}}

\begin{figure}[H]
	\centering
	\includegraphics[width=1.0\textwidth]{images/4.png}
	\caption{Wyniki 4. kwerendy}
\end{figure}

Zarówno liczba klientów/klientek, jak ich sumy wydatków, są zbliżone sobie. Jedynie w pierwszym miesiącu z racji niewielkiej liczby klientów ogólnie widać przewagę kobiet. Wniosek jest taki, że płeć nie ma wielkiego wpływu na wydatki czy na liczbę klientów danej płci.

\subsection{Kwerenda 5}

"Przeanalizuj udział sprzedanych produktów w poszczególnych podkategoriach w stosunku do całych kategorii (zarówno pod względem liczbowym jak i wartościowym)."

{\small
\begin{lstlisting}[
	language=SQL,
	showspaces=false,
	basicstyle=\ttfamily,
	numbers=left,
	numberstyle=\tiny,
	commentstyle=\color{green},
	tabsize=2
]
SELECT 
	ProductCategory.Name AS Kategoria,
	ProductSubcategory.Name AS Podkategoria,
	SUM(SalesOrderDetail.OrderQty) AS "Liczba sprzedanych produktów",
	SUM(SalesOrderDetail.LineTotal) AS "Wartość sprzedaży",
	ROUND(
		CAST(SUM(SalesOrderDetail.OrderQty) AS FLOAT) 
		/ SUM(SUM(SalesOrderDetail.OrderQty)) OVER(PARTITION BY ProductCategory.Name) 
		* 100, 2
	) AS "Udział liczbowy (%)",
	ROUND(
		CAST(SUM(SalesOrderDetail.LineTotal) AS FLOAT) 
		/ SUM(SUM(SalesOrderDetail.LineTotal)) OVER(PARTITION BY ProductCategory.Name) 
		* 100, 2
	) AS "Udział wartościowy (%)"
FROM Production.Product
INNER JOIN Production.ProductSubcategory 
	ON Product.ProductSubcategoryID = ProductSubcategory.ProductSubcategoryID
INNER JOIN Production.ProductCategory 
	ON ProductSubcategory.ProductCategoryID = ProductCategory.ProductCategoryID
INNER JOIN Sales.SalesOrderDetail 
	ON Product.ProductID = SalesOrderDetail.ProductID
GROUP BY ProductCategory.Name, ProductSubcategory.Name
\end{lstlisting}}

\begin{figure}[H]
	\centering
	\includegraphics[width=1.0\textwidth]{images/5.png}
	\caption{Wyniki 5. kwerendy}
\end{figure}

Pierwszym faktem jest to, że liczba sprzedanych produktów nie wyznacza całkowicie wartości sprzedaży. Na przykład, jeden z najliczniej sprzedających się artykułów ubioru są czapki (caps), gdzie liczba sprzedaży odpowiada za 11.28\% kategorii, ale przychód odpowiada za tylko 2.42\%.

Wśród akcesoriów uzyskuje się najwiekszy przychód przy mniejszej liczby sprzedanych egzemplarzy z Bike racks, Bike stands czy z Hydration Packs. Pomimo dużej ilości sprzedanych Bottles and Cages, przychód nie jest tak wielki.

W kategorii rowerów największy udział zarówno pod względem liczby sprzedanych produktów, jak i wartości sprzedaży mają Road Bikes. Mountain Bikes, mimo mniejszej liczby sprzedanych sztuk, generują wysoki przychód, co wskazuje na ich wyższą cenę jednostkową. Touring Bikes sprzedają się najrzadziej, jednak ich udział w wartości sprzedaży jest stosunkowo stabilny.

W kategorii odzieży najwięcej sprzedaje się Jerseys oraz Gloves. Jerseys generują największy przychód, co wynika z ich popularności oraz prawdopodobnie wyższej ceny. Z kolei Socks czy Caps, mimo wysokiej liczby sprzedanych egzemplarzy, mają niewielki udział wartościowy, co świadczy o ich niskiej cenie jednostkowej.

Wśród komponentów dominują Mountain Frames i Road Frames, które mają największy udział wartościowy przy średniej liczbie sprzedanych sztuk, co sugeruje wysoką cenę tych produktów. Z kolei Pedals, Handlebars czy Wheels sprzedają się częściej, jednak ich udział wartościowy pozostaje stosunkowo niski. Najmniejszy udział w przychodzie mają Chains, mimo, że sprzedano ich prawie 800 sztuk.

Warto podkreślić, że wysoki udział liczbowy a niski udział wartościowy nie oznacza nieudanego produktu - wręcz przeciwnie, oznacza to, że są to produkty popularne. Dochodowość zależy od marży.

\subsection{Kwerenda 5}

"Przeanalizuj udział sprzedanych produktów w poszczególnych podkategoriach w stosunku do całych kategorii (zarówno pod względem liczbowym jak i wartościowym)."

{\small
\begin{lstlisting}[
	language=SQL,
	showspaces=false,
	basicstyle=\ttfamily,
	numbers=left,
	numberstyle=\tiny,
	commentstyle=\color{green},
	tabsize=2
]
\end{lstlisting}}

\begin{figure}[H]
	\centering
	\includegraphics[width=1.0\textwidth]{images/3.png}
	\caption{Wyniki 3. kwerendy}
\end{figure}

\end{document}