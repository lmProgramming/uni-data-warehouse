\documentclass[a4paper,12pt]{article}
\usepackage[T1]{fontenc}
\usepackage[utf8]{inputenc}
\usepackage[polish]{babel}
\usepackage{color}
\usepackage{graphicx}
\usepackage{amsmath}
\usepackage{amssymb}
\usepackage{hyperref}
\usepackage{float}
\usepackage{listings}
\usepackage[backend=biber]{biblatex}

\addbibresource{bibliografia.bib}

\lstset{
  basicstyle=\ttfamily,
  columns=flexible,
  keepspaces=true,
  showstringspaces=false,
  escapeinside={(*@}{@*)},
  literate={ą}{{\k{a}}}1
           {ć}{{\'{c}}}1
           {ę}{{\k{e}}}1
           {ł}{{\l{}}}1
           {ń}{{\'{n}}}1
           {ó}{{\'{o}}}1
           {ś}{{\'{s}}}1
           {ź}{{\'{z}}}1
           {ż}{{\.{z}}}1
           {Ą}{{\k{A}}}1
           {Ć}{{\'{C}}}1
           {Ę}{{\k{E}}}1
           {Ł}{{\L{}}}1
           {Ń}{{\'{N}}}1
           {Ó}{{\'{O}}}1
           {Ś}{{\'{S}}}1
           {Ź}{{\'{Z}}}1
           {Ż}{{\.{Z}}}1
           {"}{{\textquotedbl}}1
           {'}{{\textquotesingle}}1
           {`}{{\textasciigrave}}1
           {~}{{\textasciitilde}}1
           {^}{{\textasciicircum}}1
           {_}{{\textunderscore}}1
           {|}{{\textbar}}1
           {\{}{{\textbraceleft}}1
           {\}}{{\textbraceright}}1
           {[}{{[}}1
           {]}{{]}}1
}

\title{4. sprawozdanie z laboratorium Hurtownie Danych}
\author{Mikołaj Kubś, 272662}
\date{\today}

\begin{document}

\maketitle

\begin{enumerate}
    \item DDL (Data Definition Language) - tworzenie struktur danych i schema z użyciem poleceń CREATE, ALTER, DROP
    \item DML (Data Manipulation Language) - manipulacja danymi w tabelach z użyciem poleceń INSERT, UPDATE, DELETE
    \item DCL (Data Control Language) - zarządzanie uprawnieniami, za pomocą poleceń GRANT, DENY, REVOKE
    \item DQL (Data Query Language) - pozyskiwanie danych z bazy, za pomocą polecenia SELECT
\end{enumerate}

\section{Zadanie 1 - przygotowanie schematu}

 {\small
  \begin{lstlisting}[
	language=SQL,
	showspaces=false,
	basicstyle=\ttfamily,
	numbers=left,
	numberstyle=\tiny,
	commentstyle=\color{green},
	tabsize=2
]
CREATE SCHEMA Kubs
\end{lstlisting}}

Utworzenie dedykowanego schematu pozwala na logiczne odseparowanie obiektów stworzonych na potrzeby laboratorium od pozostałych struktur bazy danych.

\section{Zadanie 2 - Tworzenie tabel wymiarów i tabeli faktów}

 {\small
  \begin{lstlisting}[
	language=SQL,
	showspaces=false,
	basicstyle=\ttfamily,
	numbers=left,
	numberstyle=\tiny,
	commentstyle=\color{green},
	tabsize=2
]

CREATE TABLE Kubs.DIM_CUSTOMER (
    CustomerID INT NOT NULL,
    FirstName NVARCHAR(50) NULL,
    LastName NVARCHAR(50) NULL,
    Title NVARCHAR(8) NULL,
    City NVARCHAR(30) NULL,
    TerritoryName NVARCHAR(50) NULL,
    CountryRegionCode NVARCHAR(3) NULL,
    [Group] NVARCHAR(50) NULL
);

CREATE TABLE Kubs.DIM_PRODUCT (
    ProductID INT NOT NULL,
    Name NVARCHAR(50) NOT NULL,
    ListPrice MONEY NULL,
    Color NVARCHAR(15) NULL,
    SubCategoryName NVARCHAR(50) NULL,
    CategoryName NVARCHAR(50) NULL,
    Weight DECIMAL(8, 2) NULL,
    Size NVARCHAR(5) NULL,
    IsPurchased BIT NULL DEFAULT 0
);

CREATE TABLE Kubs.DIM_SALESPERSON (
    SalesPersonID INT NOT NULL,
    FirstName NVARCHAR(50) NULL,
    LastName NVARCHAR(50) NULL,
    Title NVARCHAR(8) NULL,
    Gender NCHAR(1) NULL,
    CountryRegionCode NVARCHAR(3) NULL,
    [Group] NVARCHAR(50) NULL
);

CREATE TABLE Kubs.FACT_SALES (
    ProductID INT NOT NULL,
    CustomerID INT NOT NULL,
    SalesPersonID INT NULL,      
    OrderDate INT NOT NULL,        
    ShipDate INT NULL,             
    OrderQty SMALLINT NOT NULL, 
    UnitPrice MONEY NOT NULL,
    UnitPriceDiscount DECIMAL(8, 4) NOT NULL DEFAULT 0,
    LineTotal DECIMAL(19, 4) NOT NULL
);
\end{lstlisting}}

Zgodnie z poleceniem, kolumny OrderDate oraz ShipDate będą przechowywać dane typu całkowitego.
W tabeli faktów sprzedawca nie jest wymagany.
W tabeli wymiaru dla produktu kategoria i podkategoria również mogą być NULL.
Tabele DIM\_\* pełnią oczywiście role wymiarów, a FACT\_SALES jest tabelą faktów zawierającą miary dotyczące transakcji sprzedaży.

Wartości NULL w niektórych kolumnach wynikają z opcjonalności tych danych w oryginalnym źródle danych lub mogą być nieobecne (również z powodu braku połączenia z inną tabelą, np. produkt bez podkategorii nie może mieć kategorii).


\section{Zadanie 3 - wypełnianie danych}

 {\small
  \begin{lstlisting}[
	language=SQL,
	showspaces=false,
	basicstyle=\ttfamily,
	numbers=left,
	numberstyle=\tiny,
	commentstyle=\color{green},
	tabsize=2
]
INSERT INTO Kubs.DIM_CUSTOMER (
    CustomerID,
    FirstName,
    LastName,
    Title,
    City,
    TerritoryName,
    CountryRegionCode,
    [Group]
)
SELECT
    c.CustomerID,
    MIN(p.FirstName) AS FirstName,
    MIN(p.LastName) AS LastName,
    MIN(p.Title) AS Title,
    MIN(a.City) AS City,
    MIN(st.Name) AS TerritoryName,
    MIN(st.CountryRegionCode) AS CountryRegionCode,
    MIN(st.[Group]) AS [Group]
FROM Sales.Customer AS c
LEFT JOIN Person.Person AS p 
    ON c.PersonID = p.BusinessEntityID
LEFT JOIN Sales.SalesTerritory AS st 
    ON c.TerritoryID = st.TerritoryID
LEFT JOIN Person.BusinessEntityAddress bea 
    ON p.BusinessEntityID = bea.BusinessEntityID
LEFT JOIN Person.Address AS a ON bea.AddressID = a.AddressID
WHERE c.PersonID IS NOT NULL
GROUP BY c.CustomerID;

INSERT INTO Kubs.DIM_PRODUCT (
    ProductID,
    Name,
    ListPrice,
    Color,
    SubCategoryName,
    CategoryName,
    Weight,
    Size,
    IsPurchased
)
SELECT DISTINCT
    p.ProductID,
    p.Name,
    p.ListPrice,
    p.Color,
    psc.Name AS SubCategoryName,
    pc.Name AS CategoryName,
    p.Weight,
    p.Size,
    1 AS IsPurchased
FROM Production.Product AS p
INNER JOIN Sales.SalesOrderDetail AS sod 
    ON p.ProductID = sod.ProductID
LEFT JOIN Production.ProductSubcategory AS psc 
    ON p.ProductSubcategoryID = psc.ProductSubcategoryID
LEFT JOIN Production.ProductCategory AS pc 
    ON psc.ProductCategoryID = pc.ProductCategoryID;

INSERT INTO Kubs.DIM_SALESPERSON (
    SalesPersonID,
    FirstName,
    LastName,
    Title,
    Gender,
    CountryRegionCode,
    [Group]
)
SELECT
    sp.BusinessEntityID AS SalesPersonID,
    p.FirstName,
    p.LastName,
    p.Title,
    e.Gender,
    st.CountryRegionCode,
    st.[Group]
FROM Sales.SalesPerson AS sp
INNER JOIN Person.Person AS p 
    ON sp.BusinessEntityID = p.BusinessEntityID
INNER JOIN HumanResources.Employee AS e 
    ON sp.BusinessEntityID = e.BusinessEntityID
LEFT JOIN Sales.SalesTerritory AS st 
    ON sp.TerritoryID = st.TerritoryID;

INSERT INTO Kubs.FACT_SALES (
    ProductID,
    CustomerID,
    SalesPersonID,
    OrderDate,
    ShipDate,
    OrderQty,
    UnitPrice,
    UnitPriceDiscount,
    LineTotal
)
SELECT
    sod.ProductID,
    soh.CustomerID,
    soh.SalesPersonID,
    DATEPART(YEAR, soh.OrderDate) * 10000 + 
    DATEPART(MONTH, soh.OrderDate) * 100 + 
    DATEPART(DAY, soh.OrderDate) AS OrderDate,
    DATEPART(YEAR, soh.ShipDate) * 10000 + 
    DATEPART(MONTH, soh.ShipDate) * 100 +
    DATEPART(DAY, soh.ShipDate) AS ShipDate,
    sod.OrderQty,
    sod.UnitPrice,
    sod.UnitPriceDiscount,
    sod.LineTotal
FROM Sales.SalesOrderDetail AS sod
INNER JOIN Sales.SalesOrderHeader AS soh 
    ON sod.SalesOrderID = soh.SalesOrderID;
\end{lstlisting}}

\begin{figure}[H]
    \centering
    \includegraphics[width=1.0\textwidth]{images/3_customers.png}
    \caption{Pierwsze 3 wiersze DIM\_CUSTOMER i liczba wierszy - 19119}
\end{figure}

\begin{figure}[H]
    \centering
    \includegraphics[width=1.0\textwidth]{images/3_products.png}
    \caption{Pierwsze 3 wiersze DIM\_PRODUCT i liczba wierszy - 266}
\end{figure}

\begin{figure}[H]
    \centering
    \includegraphics[width=1.0\textwidth]{images/3_sale_people.png}
    \caption{Pierwsze 3 wiersze DIM\_SALESPERSON i liczba wierszy - 17}
\end{figure}

\begin{figure}[H]
    \centering
    \includegraphics[width=1.0\textwidth]{images/3_sales.png}
    \caption{Pierwsze 3 wiersze FACT\_SALES i liczba wierszy - 121317}
\end{figure}

Zgodnie z poleceniem wypełniono tabele danymi z bazy danych AdventureWorks2014.

Ponieważ analizy będą dotyczyć sprzedaży produktów, pominięto wszystkie produkty, które nigdy nie zostały kupione. Okazało się, że wszystkie produkty, które nie miały kategorii lub podkategorii, nigdy nie były kupione.

Potem okazało się, że jest parę wierszy dla klientów z tym samym CustomerID. W takim wypadku wstawiono tylko jeden wiersz.

Liczby wierszy:

DIM\_CUSTOMER - 19119

DIM\_PRODUCT - 266

DIM\_SALESPERSON - 17

FACT\_SALES - 121317

\section{Zadanie 4 - Więzy integralności}

\subsection{Definiowanie kluczy głównych i obcych}

Poniższy kod SQL dodaje klucze główne do tabel wymiarów oraz klucze obce do tabeli faktów, łącząc ją z wymiarami.

    {\small
        \begin{lstlisting}[
    language=SQL,
    showspaces=false,
    basicstyle=\ttfamily,
    numbers=left,
    numberstyle=\tiny,
    commentstyle=\color{green},
    tabsize=2
]
ALTER TABLE Kubs.DIM_CUSTOMER
ADD CONSTRAINT PK_DIM_CUSTOMER PRIMARY KEY (CustomerID);

ALTER TABLE Kubs.DIM_PRODUCT
ADD CONSTRAINT PK_DIM_PRODUCT PRIMARY KEY (ProductID);

ALTER TABLE Kubs.DIM_SALESPERSON
ADD CONSTRAINT PK_DIM_SALESPERSON PRIMARY KEY (SalesPersonID);

ALTER TABLE Kubs.FACT_SALES
ADD CONSTRAINT FK_FACT_SALES_DIM_CUSTOMER
FOREIGN KEY (CustomerID) REFERENCES Kubs.DIM_CUSTOMER(CustomerID);

ALTER TABLE Kubs.FACT_SALES
ADD CONSTRAINT FK_FACT_SALES_DIM_PRODUCT
FOREIGN KEY (ProductID) REFERENCES Kubs.DIM_PRODUCT(ProductID);

ALTER TABLE Kubs.FACT_SALES
ADD CONSTRAINT FK_FACT_SALES_DIM_SALESPERSON
FOREIGN KEY (SalesPersonID) 
    REFERENCES Kubs.DIM_SALESPERSON(SalesPersonID);
\end{lstlisting}
    }

\subsection{Testowanie więzów integralności}

Poniższe instrukcje \texttt{INSERT INTO} mają na celu sprawdzenie działania zdefiniowanych kluczy głównych i obcych. Oczekujemy, że próby wstawienia niepoprawnych danych zakończą się błędami przechwyconymi w blokach \texttt{CATCH}.

{\small
\begin{lstlisting}[
    language=SQL,
    showspaces=false,
    basicstyle=\ttfamily,
    numbers=left,
    numberstyle=\tiny,
    commentstyle=\color{green},
    tabsize=2
]
PRINT 'Test 1: Próba naruszenia PK w DIM_CUSTOMER';
BEGIN TRY
    INSERT INTO Kubs.DIM_CUSTOMER (CustomerID, FirstName, LastName)
    VALUES (11000, 'Test', 'DuplicatePK');

    PRINT 'BŁĄD - duplikat PK';
END TRY
BEGIN CATCH
    PRINT 'SUKCES';
    PRINT ERROR_MESSAGE();
END CATCH
GO

PRINT 'Test 2: Próba naruszenia FK (nieistniejący ProductID) w FACT_SALES';
BEGIN TRY
    INSERT INTO Kubs.FACT_SALES (
        ProductID, CustomerID, SalesPersonID, OrderDate, ShipDate,
        OrderQty, UnitPrice, UnitPriceDiscount, LineTotal
    ) VALUES (
        -999,
        11000,
        NULL,
        20250101, 20250102, 1, 10.0, 0, 10.0
    );

    PRINT 'BŁĄD: Nieistniejące ProductID';
END TRY
BEGIN CATCH
    PRINT 'SUKCES';
    PRINT ERROR_MESSAGE();
END CATCH
GO

PRINT 'Test 3: Próba naruszenia FK (nieistniejący CustomerID) w FACT_SALES';
BEGIN TRY
    INSERT INTO Kubs.FACT_SALES (
        ProductID, CustomerID, SalesPersonID, OrderDate, ShipDate,
        OrderQty, UnitPrice, UnitPriceDiscount, LineTotal
    ) VALUES (
        776,
        -999,
        NULL,
        20240103, 20240104, 2, 20.0, 0, 40.0
    );

    PRINT 'BŁĄD: nieistniejący CustomerID';
END TRY
BEGIN CATCH
    PRINT 'SUKCES';
    PRINT ERROR_MESSAGE();
END CATCH
GO

PRINT 'Koniec testów'
\end{lstlisting}
}

\begin{figure}[H]
    \centering
    \includegraphics[width=1.0\textwidth]{images/4.png}
    \caption{Wykonanie testów więzów integralności}
\end{figure}

Testowe instrukcje zakończyły się przechwyceniem błędów związanych z naruszeniem więzów integralności.

\section{Zadanie 5 - tworzenie kostki}

W środowisku Visual Studio utworzono nowy projekt typu "Analysis Services Multidimensional and Data Mining Project" o nazwie "FirstCube".

Następnie skonfigurowano źródło danych (\texttt{Data Source}), wskazując na bazę \texttt{AdventureWorks} i wykorzystując uwierzytelnianie systemu Windows. Informacje o personifikacji (\texttt{Impersonation Information}) ustawiono na użycie konta usługi (\texttt{Use the service account}), co wymagało nadania odpowiednich uprawnień (\texttt{db\_datareader}) temu kontu w bazie SQL Server.
Pozwoliło to w zadaniu 6. na poprawne wykonanie \texttt{Process...}
{\small
    \begin{lstlisting}[
    language=SQL,
    showspaces=false,
    basicstyle=\ttfamily,
    numbers=left,
    numberstyle=\tiny,
    commentstyle=\color{green},
    tabsize=2
    ]
    ALTER ROLE db_datareader ADD MEMBER [NT Service\MSSQLServerOLAPService];
    \end{lstlisting}
}

Utworzono widok źródła danych (\texttt{Data Source View}), dodając tabele wymiarów oraz tabelę faktów.

Za pomocą kreatora utworzono nową kostkę o nazwie \texttt{SalesCubeView}, bazując na istniejących tabelach z widoku źródła danych:
\begin{itemize}
    \item Tabela \texttt{FACT\_SALES} została wskazana jako tabela faktów (grupa miar).
    \item Wybrano miary: \texttt{OrderQty}, \texttt{UnitPriceDiscount} oraz \texttt{LineTotal}. Pozostałe kolumny tabeli faktów (\texttt{ProductID}, \texttt{CustomerID}, \texttt{SalesPersonID}, \texttt{OrderDate}, \texttt{ShipDate}) nie są miarami, ponieważ pełnią rolę kluczy obcych do wymiarów lub przechowują informacje opisowe (daty), których agregacja numeryczna (np. suma dat) nie ma sensu biznesowego. Miary reprezentują wartości ilościowe lub pieniężne, które można agregować (sumować, liczyć, uśredniać itp.).
    \item Tabele \texttt{DIM\_CUSTOMER}, \texttt{DIM\_PRODUCT}, \texttt{DIM\_SALESPERSON} zostały wybrane jako wymiary kostki.
\end{itemize}

Po utworzeniu kostki dokonano edycji wymiarów, dodając do nich odpowiednie atrybuty.

\begin{figure}[H]
    \centering
    \includegraphics[width=0.8\textwidth]{images/5.png}
    \caption{Struktura kostki w edytorze Visual Studio}
\end{figure}

\section{Zadanie 6 - Uruchomienie kostki}

Jako serwer docelowy (\texttt{Target Server}) ustawiono \texttt{localhost} i podano nazwę docelowej bazy danych SSAS Lab04.

\begin{figure}[H]
    \centering
    \includegraphics[width=1.0\textwidth]{images/6_deployment_properties.png}
    \caption{Ustawienia konfiguracji wdrożenia}
\end{figure}

Następnie wdrożono projekt za pomocą opcji \texttt{Build -> Deploy Solution}. Wdrożenie zakończyło się sukcesem.

\begin{figure}[H]
    \centering
    \includegraphics[width=1.0\textwidth]{images/6_deployment_success.png}
    \caption{Udany deployment}
\end{figure}

Kolejnym krokiem było przetworzenie kostki. Kliknięto prawym przyciskiem na kostkę w \texttt{Solution Explorer} i wybrano opcję \texttt{Process...} (jedną z 4). Proces przetwarzania również zakończył się powodzeniem (choć dopiero po nadaniu poprawnych uprawnień).

\begin{figure}[H]
    \centering
    \includegraphics[width=1.0\textwidth]{images/6_success.png}
    \caption{Potwierdzenie pomyślnego przetworzenia kostki}
\end{figure}

\section{Zadanie 7 - Proste raporty}

Do analizy danych zawartych w kostce wykorzystano program MS Excel. Połączono się z przetworzoną kostką Analysis Services, wybierając opcję w GUI Visual Studio wewnątrz SalesCubeView -> Browser -> Analyze in Excel.

Przeprowadzono przykładowe analizy, tworząc tabele przestawne i wykresy na ich podstawie:
\begin{itemize}
    \item Analiza sumy sprzedaży (\texttt{Line Total}) według hierarchii geograficznej klientów (Grupa -> Kod Państwa -> Nazwa Terytorium).
    \item Analiza liczby sprzedanych sztuk (\texttt{Order Qty}) i sumę udzielonych zniżek (\texttt{Unit Price Discount}) według sprzedawców.
    \item Analiza sumy sprzedaży (\texttt{Line Total}) w zależności od koloru produktu (\texttt{Color}).
\end{itemize}

\begin{figure}[H]
    \centering
    \includegraphics[width=1.0\textwidth]{images/7_1.png}
    \caption{Analiza sumy sprzedaży (\texttt{Line Total}) według hierarchii geograficznej klientów (Grupa -> Kod Państwa -> Nazwa Terytorium).}
\end{figure}

\begin{figure}[H]
    \centering
    \includegraphics[width=1.0\textwidth]{images/7_2.png}
    \caption{Analiza liczby sprzedanych sztuk (\texttt{Order Qty}) i sumę udzielonych zniżek (\texttt{Unit Price Discount}) według sprzedawców.}
\end{figure}

\begin{figure}[H]
    \centering
    \includegraphics[width=1.0\textwidth]{images/7_3.png}
    \caption{Analiza sumy sprzedaży (\texttt{Line Total}) w zależności od koloru produktu (\texttt{Color})}
\end{figure}

\printbibliography

\end{document}